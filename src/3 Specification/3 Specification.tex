Student provides a list of relevant requirements for the final project deliverable that can be used to measure the success of the project. Specification points are suitably prioritised with additional requirements over the main aim given as “stretch goals” and grouped in a suitable way for the project (for example splitting hardware and software requirements).

Specification points are all quantitatively justified with respect to the project aim(s) and wider literature (e.g. H&S, legal requirements, engineering standards, best practice, etc). With inclusion of suitable references.

There is clear assessment of the expected challenge posed by each specification point.

Each requirement includes a succinctly clear explanation of how it will be tested including the criteria for success.

Specifications are shown to be both attainable (with the facilities available to the student) and realistic (in terms of the time frame of the project).  

Clear ambition to achieve specification beyond commercial-off-the-shelf or state-of-the-art research-grade system. There is evidence that such ambition is realisable. 