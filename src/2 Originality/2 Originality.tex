Student provides a suitable explanation of the originality of the project, highlighting where it fits within commercial/scientific context along with differentiating from existing intellectual property.
Student provides a suitable explanation of the originality of the project and attempts to demonstrate the need for this through assessment of the current commercial/scientific context.

The need for the project is clearly identified through assessment of both the wider background and a comparative look at existing IP/current practice. There is clear differentiation from existing IP/current practice.

Student acknowledges the limitations their project will have when compared to other work (e.g. less comprehensive testing facilities, only prototype manufacturing equipment). These are both discussed and justified with relation to originality. Student can quantify and extrapolate the impact of the limitations to their project specification.

Student explains project originality not only in the final deliverable, but also in the development approach.

Multifaceted novelty combining two or more of: new technique/approach/algorithm, interdisciplinary approach, translation from one technology area to others, low-cost solution with comparable performance to current practice.