Student provides a suitable development process for the project, highlighting some key tools and components that will be used within each step, and outlining the obvious limitations related to these. Selected tools are suitably justified with respect to the project aim/specification/ /time/etc.

Discussion of the implications of inter-dependencies between different steps.

Each step of the methodology is assessed in terms of the challenge posed to the student based on their individual experience and skill set. Additional tasks related to these are explained (e.g. use of a piece of software may require additional tasks to become familiar).

Limitations are explained and assessed with relation to the project development (e.g. using a licenced software may incur cost but be beneficial if the student already has experience).

Student effectively combines/synthesizes/translates existing methodologies leading to a bespoke solution which meets the specifications and accounts for all limitations.